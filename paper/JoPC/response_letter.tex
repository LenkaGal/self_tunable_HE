\documentclass[a4paper,10pt]{article}

\usepackage{epsfig}
\usepackage{amsmath}
\usepackage{amssymb}
\usepackage{a4wide}
\usepackage{multicol}
\usepackage[ansinew]{inputenc}
\usepackage{color}
\usepackage{url}
%\usepackage{floatflt}
\usepackage{multirow}
\usepackage{rotating, graphicx}


\newcommand{\opt}{^{\star}} % \opt
\newcommand{\tr}{\intercal} % transpose



% RESPONSE LETTER SYNTAX:
% =====================================================
\newcommand{\change}[1]{\textcolor{red}{#1}}

\newcommand{\comment}[1]{
	\begin{itemize}
		\item \textbf{Comment:}\\ \sf{#1}
	\end{itemize}
}

\newcommand{\answer}[1]{
	\begin{itemize}
		\item[] \textbf{Answer:}\\ #1
	\end{itemize}
}

\newcommand{\pointout}[1]{\textcolor{blue}{#1}}
\definecolor{YJ}{rgb}{0.0,0.5,0.0}

% =====================================================


\newcommand{\diam}[1]{\mathrm{diam}\left(#1\right)}

% \newcommand {\mm}[1]{\textcolor{blue}{#1}}

%opening
\title{Response to the review comments}
\author{}
\date{}

\sloppy
\begin{document}
	
	\maketitle
	
	We would like to thank the reviewers for the constructive and thoughtful comments on our manuscript.
	We revised the document based on these comments and we provide point-to-point answers below.
	All changes relative to our initial submission are highlighted in \change{red}.
	
	\section*{Answer to Editor}
	We would like to thank the Editor for summarizing substantial and high-quality review comments, which helped us to improve our manuscript's quality. We revised the paper based on the reviewers' comments and incorporated the suggestions into our manuscript. We reformulated or added the corresponding statements to clarify the ideas. 
	
	\section*{Answer to Reviewer 1}
	\comment{The paper develops a method to automatically tune explicit MPC controllers, and demonstrates application of the proposed approach on a laboratory scale heat exchanger system. Overall, the paper is well written and the presented ideas and experimental results are clear. However, the following points should be addressed before publication: \\
	
	The main motivation that a practitioner may want different controller tunings based on operating scenarios and different sizes of reference changes/disturbances does not clearly stand out in the introduction section. The initial few paragraphs can be revised to clearly emphasize the need of a "self-tunable" MPC controller in practice.
	}

	\answer{
		In order to clearly emphasize the need of a self-tunable technique, we revised the Introduction section in a following way:
		\begin{quote}
			The paper [27] pushes the idea of tunable explicit MPC further and deals with the issues of practical industrial-oriented implementation.
		\end{quote}
	}


	\comment{
		The authors state, "The lower computational complexity makes the explicit MPC more suitable for a practical industrial implementation." in the introduction section. It should be noted, however, that the explicit MPC suffers from the curse of dimensionality, and the number of polytopic regions in the PWA control law can grow exponentially with the size of states, control inputs, and the MPC forecasting horizon. Thereby, making it not practically implementable for a large class of industrial systems.
	}

	\answer{
		We thank the reviewer for an important comment. We reformulated the sentence as follows:
		\begin{quote}
			\change{If the MPC optimization problem can be pre-solved explicitly, the lower online computational complexity makes the explicit MPC more suitable for a practical industrial implementation.}
		\end{quote}
	}
	
	\comment{
		The main idea behind the self-tunable approximate explicit MPC controller is that two extreme controllers are designed with aggressive and slow tunings called the "boundary controllers", and a linear interpolation of those two controllers is taken to produce the final control input. How can this approach be applied to multivariable interacting systems with multiple control inputs?
	}

	
	\comment{
		In the final simulation studies presented, the closed-loop performance of the self-tuned approximate MPC is compared only with the two extreme boundary controllers. However, a moderately tuned (E.g, Qy=500-600) "optimal" MPC controller may also provide a reasonable and potentially better performance than the two boundary controllers. I encourage the authors to comment on its performance and elaborate whether the control improvements reported in Table 2 still remain.
	}

	\answer{
		We thank the reviewer for the question and comment. If a moderately tuned optimal MPC also provided a reasonable or potentially better performance than the two boundary controllers, the self-tunable technique would be out of scope for such control application. Nevertheless, there are situations, when using only one controller with a constant setup leads to just ''satisfactory'' control results, i.e., the reference value is achieved, but with worse control performance (e.g. high overshoots or settling time). When working on our case study results, different setups of penalty matrices were explored, also $Q_\mathrm{y} = 500$. In this scenario, the control performance was comparable with the tuned controller for the first two reference step changes (tracking the reference upwards). However, when the reference was tracked downwards, the undershoots similar to lower boundary controller, i.e., $Q_\mathrm{y} = 100$ were present. Therefore, it supports the fact that the general aim of the control, i.e., achieving the reference value, is satisfied, but the control performance could be improved by including the benefits of two boundary controllers.   
	}


	\comment{
		The performance of the two boundary controllers shown in Figures 6 and 7 are also depicted in Figures 10 and 11. So some of the figures can be removed for a concise presentation of the results.
	}

	\answer{
		We removed the Figure 6 and 7 in order to improve the concise presentation of the control results.
	}
	
	\comment{
		Unmeasured disturbances are encountered in a large-class of industrial applications, and treating such disturbances is almost a required feature in the control algorithm to be considered for practical deployment. I recommend the authors to comment on how can the approach be applied to systems with unmeasured disturbances, or what complications may arise.
	}
	
	\answer{
		We thank the reviewer for this important comment. We revisited the Conclusion section containing the comments on disturbances and incorporated also the answer on unmeasured disturbances. (Na zvazenie: As the thoughts and ideas remind discussion more than conclusion, this part was transferred to the section Results and discussion.)
		\begin{quote}
			Note, this strategy relies on a proper design of the two boundary controllers.
			In case a non-negligible disturbance occurs, both boundary controllers
			should be able to solve a disturbance rejection problem, as the final
			value of the manipulated variables is interpolated between them. \change{It is also possible to include the disturbance into the design process of the MPC controller. In such case, e.g., a robust MPC strategy (REF?) would be beneficial, which would lead to more conservative control actions in order to create space for a disturbance. Nevertheless, if the problem is not large-scale and it is possible to obtain the explicit solution, it is still possible to interpolate between the control actions from the robust controllers.} The future work will focus on incorporating also the measurable disturbances into the self-tuning strategy. (Really?)
		\end{quote}	
	}


	
	\section*{Answer to Reviewer 2}	

	\comment{
		A self-tuning explicit MPC scheme is presented and applied to a heat exchange process. The main contribution above the authors other work appears to be the self-tuning aspect. Overall, the paper would greatly benefit from a thorough revision, clarifying ambiguous definitions, discussions, and descriptions. Also, the experimental setup should be clarified, as important dynamics that seem relevant (hot stream dynamics) appear to be ignored, raising doubts about the repeatability and therefore, conclusions drawn from the experiments. My specific comments are listed below.
		
		In the abstract, define what the term "properly tune the controller" means.
	}

	\answer{
		We understand that this formulation might sound vague, so we reformulated the corresponding sentence as follows:
		\begin{quote}
			This paper provides a novel self-tunable control policy that does not require any interventions of the control engineer during operation in order to \change{retune the controller}.
		\end{quote}	
	}

	\comment{
		Optimize the phrasing of the second paragraph of the introduction. The phrasing makes it sounds like heat exchangers are energy demanding, but I assume the intended meaning is that the utility generation for heating/cooling is energy-demanding.
	}

	\answer{
		We thank the Reviewer for this important comment. We utilized the Reviewer's suggested formulation in a following way:
		\begin{quote}
			\change{Simultaneously, the utility generation for heating or cooling is energy-demanding.}
		\end{quote}	
	}

	\comment{
		Define what is meant by asymmetric behavior.
	}

	\answer{
		We revisited the Introduction section in order to define the asymmetric behavior at the place of its first occurrence.
		\begin{quote}
			From the control viewpoint, the controller design taking into account the nonlinear and asymmetric behavior \change{(i.e. different dynamics (behavior?) when the temperature is rising or decreasing)} of the heat exchangers is a challenging task [4].
		\end{quote}	
	}


	\comment{
		What are the assumptions on the penalty matrices? Include an explicit statement of these assumptions Section 2.1 (e.g., must be diagonal matrices).
	}

	\answer{
		We revisited the Section 2.1 in order to define the assumptions on the penalty matrices. The corresponding part reads as:
		\begin{quote}
			The sets $\mathcal{U} \subseteq \mathbb{R}^{n_{\mathrm{u}}}$, $\mathcal{Y} \subseteq \mathbb{R}^{n_{\mathrm{y}}}$ are convex polytopic sets of physical constraints on inputs and outputs, respectively. These sets include the origin in their strict interiors. The penalty matrix $Q_\mathrm{y} \in \mathbb{R}^{n_{\mathrm{y}} \times n_{\mathrm{y}}}$ \change{$\succeq 0$} penalizes the squared control error, i.e., the deviation between the output and output reference value $y_\mathrm{ref}$. The matrix $R \in \mathbb{R}^{n_{\mathrm{u}} \times n_{\mathrm{u}}}$ \change{$\succ 0$} penalizes the squarred value of control inputs. 
			%
			The value of integrator is also penalized in the cost function with the penalty matrix $Q_\mathrm{I} \in \mathbb{R}^{n_{\mathrm{y}} \times n_{\mathrm{y}}}$ \change{$\succeq 0$}. \change{All the penalty matrices are considered to be diagonal due to the applicability of the tunable explicit MPC approach.}
			The parameter $\theta \in \Theta$ in Eq.~(1f) represents the initial condition of the optimization problem for which it is parametrically pre-computed. 
		\end{quote}	
	}

	\comment{
		Similarly, for the two boundary controllers explain the assumptions on the penalty matrices. Clearly, if the "upper" boundary matrices are equivalent to the "lower" boundary matrices multiplied by a scalar, there will be no difference between the solutions obtained by solving either controller. Perhaps, this is what meant by statement: "The boundary explicit controllers have the same structure and setup, except for one of the penalty matrices - the tuned one", but this statement can be further clarified (which matrix is the tuned one referring to?).
	}

	\answer{
		To explain the assumptions on the penalty matrices and specify the tuned matrix in the corresponding paragraph of the Section 2.2 in the following way:
		\begin{quote}
			The idea of approximated tunable explicit MPC comes from the work ~\cite{Klauco_tunable}, where the control action is calculated based on linear interpolation between two boundary control actions. These control actions result from evaluating two boundary explicit MPCs. The boundary explicit controllers have the same structure and setup, except for one of the penalty matrices -- the tuned one. \change{Based on the specific control application, any penalty matrix can be chosen as the tuned parameter, i.e., this approach is applicable for any tuning parameter. The boundary penalty matrices follow the assumptions on the penalty matrices from Section 2.1 and are diagonal matrices such that $\lambda_{i,\mathrm{L}} \le \lambda_{i,\mathrm{U}}$, $\forall i = 1,\dots,s$, where $\lambda$ denotes the vector of eigenvalues of the penalty matrix, $s$ is the rank of the tuned penalty matrix, and $L$, $U$ denote the lower and upper boundary setup respectively. }
		\end{quote}	
	}

	\comment{
		Please clarify the last sentence on page 7, extending into page 8, which describes the tuning process as systematic, but the description appears to be ad hoc and depend on the system.
	}

	\comment{
		Does the method only support SISO or MIMO system with decoupled pairs of inputs/outputs? This seems to be suggested by the statement made on page 9. If so, this should be stated explicitly in the problem setup.
	}

	\comment{
		How is the aggressiveness of the controller being defined?
	}

	\comment{
		What are the properties of the reference values selected? Are they reachable?
	}

	\answer{
		We would like to divide the answer in two parts as the reference value is present in MPC optimization problem in Eq.~(1) and also in the calculation process of $\rho$ in Eq.~(7) and Eq.~(9). Regarding the MPC optimization problem, the reference signal is a parameter which is usually constrained such that it covers the operating range of the corresponding control application. Therefore, during the control, the reference can be set only as a value from the feasible set. If the MPC problem is precomputed also for non-reachable references, e.g., because it is not clear a priori which reference values will be set during control, this would ''only'' lead to larger explicit solution of MPC because of larger range of reference signal. Regarding the calculation of parameter $\rho$, the setup of reference value is more strict, i.e., the value should be reachable and from the operating range to ensure that $0 \le \rho \le 1 $. 
		\begin{quote}
			TODO
		\end{quote} 
		
	}

	\comment{
		In Eq. 7, there is a missing time dependence on rho and the "(k)" in $y_\mathrm{ref}$ should not be a subscript.
	}

	\answer{
		We thank the Reviewer for a careful review. The equation was corrected as follows:
		\begin{quote}
			\change{
			\begin{eqnarray*}				
				\rho (k) = \frac{\vert y_{\mathrm{ref}}(k) \vert}{d_{\max}}.
			\end{eqnarray*}
		}
		\end{quote} 
	
	}

	\comment{
		Please clarify the role of rho and dmax. What is the symbol $\lvert.\rvert$ mean for a vector (element-wise absolute value)? It seems that dmax would need to be inherently conservative, in the sense that it is the maximum magnitude of all possible references values. If this is the case, it would seem that rho would mostly be less than 1. On the other hand, if dmax is not the maximum magnitude of all possible reference, there is a possibility of rho $>$ 1. What happens in this case? Why is it appropriate to make rho closer to 1 for references with large magnitudes.
	}

	\comment{
		At the end of Section 2, it is mentioned that the positivity or negativity of the reference change could be considered. What does this mean? A precise mathematical description would be helpful.
	}

	\comment{
	 	Given the lack of clear definition of asymmetric behavior, the need for a modified self tuning technique (Section 3.2) is a bit unclear. A clear description of its need in the beginning of Section 3.2 would be helpful.
	}

	\comment{
		In Definition 3.1, is the domain of gamma indeed R?
	}

	\answer{
		The parameter $\gamma$ depends on the value of $\rho$, which is scalar. From this it follows, that the domain of $\gamma$ is $\mathbb{R}$.
	}	

	\comment{
		Figure 4 seems out of place (should be after figure 1).
	}

	\answer{
		We thank the reviewer for this attentive comment. We corrected the order of Figures.
	}

	\comment{
		A diagram or schematic of the heat exchange process would be helpful. Also, referring to the cold water stream as the feed when the outlet temperature is regulated is a bit confusing.
	}

	\comment{
		How was it determined that Qy has the most significant effect in the heat exchange process? What constitutes "significant effect"?
	}

	\answer{
		\begin{quote}
			The penalty matrices of the problem in Eq.~(1) were systematically tuned and the corresponding control was implemented on the laboratory heat exchanger for each of the considered explicit MPC setups. 
			First, the aim of tuning was to determine, which penalty matrix is the most suitable for real-time tuning. \change{Based on the set of experimentally collected data, the penalty matrix $Q_\mathrm{y}$ was chosen as the tuned one.} Based on the set of experimentally collected data, the most significant effect on the control trajectories had tuning the penalty matrix $Q_\mathrm{y}$, while still preserving a satisfactory control performance, i.e., without steady-state control error and significant oscillations around the reference value. Next, the boundary values of the tunable matrix $Q_\mathrm{y}$ were tuned as $Q_\mathrm{y, L}$ = 100 and $Q_\mathrm{y, U}$ = 1\,000.
		\end{quote}
	}

	\comment{
		In the heat exchange process, the dynamics of the hot stream appears to be neglected. Why is this the case? I am not sure about the explanation given on why different steady-state inputs are observed. The flow rate of the hot stream is adjusted so the cold stream outlet temperature is maintained at the reference value is done. It seems, though not shown, that the hot stream inlet temperature is maintained to be constant. Based on the results, it would seem that the steady-state hot stream temperature is free to vary. What about the cold stream inlet temperature? Please given the trajectories of all these variables, or at least, the temperatures. What is the impact of the repeatability of the experiments given that the hot stream temperature is allowed to vary? In any case, the metrics presented in Table 1 are reported for one experiment per scenario. I would encourage computing these metrics for several experiments to ensure repeatability of the experiments and make the results more convincing.
	}

	\answer{
		 		
	}	

	
	\bibliographystyle{plain}
	\bibliography{references}
	
\end{document}


