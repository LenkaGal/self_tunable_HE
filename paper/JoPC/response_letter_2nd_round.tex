\documentclass[a4paper,10pt]{article}

\usepackage{epsfig}
\usepackage{amsmath}
\usepackage{amssymb}
\usepackage{a4wide}
\usepackage{multicol}
\usepackage[ansinew]{inputenc}
\usepackage{color}
\usepackage{url}
%\usepackage{floatflt}
\usepackage{multirow}
\usepackage{rotating, graphicx}


\newcommand{\opt}{^{\star}} % \opt
\newcommand{\tr}{\intercal} % transpose



% RESPONSE LETTER SYNTAX:
% =====================================================
\newcommand{\change}[1]{\textcolor{red}{#1}}

\newcommand{\comment}[1]{
	\begin{itemize}
		\item \textbf{Comment:}\\ \sf{#1}
	\end{itemize}
}

\newcommand{\answer}[1]{
	\begin{itemize}
		\item[] \textbf{Answer:}\\ #1
	\end{itemize}
}

\newcommand{\pointout}[1]{\textcolor{blue}{#1}}
\definecolor{YJ}{rgb}{0.0,0.5,0.0}

% =====================================================


\newcommand{\diam}[1]{\mathrm{diam}\left(#1\right)}

% \newcommand {\mm}[1]{\textcolor{blue}{#1}}

%opening
\title{Response to the review comments}
\author{}
\date{}

\sloppy
\begin{document}
	
	\maketitle
	
	We thank the Reviewers for all comments and suggestions.
	The document was revised according to these comments and we provide point-to-point answers below.
	All changes relative to our initial submission are highlighted in \change{red}. 
	
	\section*{Answer to Editor}
	The authors would like to thank the Editor for collecting the review comments. We revised the paper based on the Reviewers' comments, and incorporated the suggestions
	into our manuscript.

	
	\section*{Answer to Reviewer 1}
	\comment{ The authors have not addressed some of my comments in the previous submission thoroughly. I have the following follow-up comments:\\ 
	
	 The control performance with Qy = 500 is not quantified, and whether the performance benefits of the proposed method reported in Table 2 still remain is not clarified. Since Qy =500 would give a moderately tuned controller, it can provide a better overall performance than the two boundary controllers chosen with "extreme" tuning.
	}

	\answer{		
		We revisited Table~\ref{tab:control_performance} to evaluate the control performance criteria.
		The values highlighted in \textbf{bold} represent the scenarios when the controller with the investigated setup $Q_\mathrm{y} = 500$ performed as the best one, excluding the self-tunable MPC. It can be seen, that the values are the best in 5 out of 12 performance criteria. On the contrary, the self-tuned controller acquired 10 best values out of 12 performance criteria thanks to the tuning based on the varying operating conditions. 
		
		When excluding the boundary controllers and comparing only the self-tunable controller with the setup $Q_\mathrm{y} = 500$, the moderate controller outperforms the self-tunable MPC in half of the performance criteria. It happens mostly in tracking the reference value upwards. Nevertheless, when the reference changes downwards, the self-tuned controller becomes beneficial. Therefore, it supports choosing the tunable control technique. 
		
		\textcolor{blue}{Pozn.: Zvazme teraz dva scenare - bud toto o 500vke zapracujem nejako do clanku, alebo to odbavime nasledovne:}
		
		To ensure a good readability of the paper, we incline not to branch the case study also in the direction of the moderate setup, i.e., $Q_\mathrm{y} = 500$. We prefer to emphasize the consideration of the self-tuning method based on the revised comment from the first round of review:
		
		\begin{quote}
			Obviously, if there exists a well-tuned ``universal'' controller that satisfies the requirements on the control performance in the whole range of the considered operating conditions, then the implementation of the self-tuning procedure is out of scope for such control application. Nevertheless, in numerous practical situations, using only one controller with a constant setup leads to poor or just ``satisfactory'' control results, i.e., the reference value is achieved, but with worse control performance, e.g., leading to high overshoots or settling times. When working on our laboratory case study, a set of different setups of penalty matrices was investigated. In every control scenario, the setup was beneficial only in some working conditions (tracking the reference upwards or downwards). Therefore, the closed-loop control performance is improved by introducing the benefits of the self-tuning method based on the two boundary MPC controllers.
		\end{quote}
		
	\begin{table}[h!]
		\begin{center}
			\caption{Control performance criteria.}
			\label{tab:control_performance}
			\begin{tabular}{c|c|c|c|c} 
				Reference step change & $Q_\mathrm{y}$ & SSE [$^{\circ}\mathrm{C}^2$\,s] & $\sigma_{\mathrm{max}}$\,[\%] & $t_{\epsilon}$\,[s]  \\
				\hline
				\multirow{4}{*}{ 35\,$^{\circ}$C $\rightarrow$ 45\,$^{\circ}$C } & 1\,000 & 714 & 33.5 & 16.5 \\
				& 100 & 867 & 16.7 & 12.5 \\ 
				& 500 & \textbf{583} & 17.6 & \textbf{7.5} \\
				& self-tuned & 678 & 15.2 & 9.5  \\ 
				\hline
				\multirow{4}{*}{ 45\,$^{\circ}$C $\rightarrow$ 50\,$^{\circ}$C } & 1\,000 & 365 & 47.2 & 5 \\
				& 100 & 606 & 23.3 & 26.5  \\ 
				& 500 & \textbf{168} & \textbf{18.2} & 6.5 \\
				& self-tuned & 248 & 19.1 & 9.5  \\ 
				\hline
				\multirow{4}{*}{ 50\,$^{\circ}$C $\rightarrow$ 45\,$^{\circ}$C } & 1\,000 & 245 & 18.9 & 6.5  \\
				& 100 & 398 & 79.6 & 31  \\ 
				& 500 & \textbf{211} & 22.3 & 8 \\
				& self-tuned & 186 & 24.6 & 6.5  \\ 
				\hline
				\multirow{4}{*}{ 45\,$^{\circ}$C $\rightarrow$ 35\,$^{\circ}$C } & 1\,000 & 1\,024 & 18.4 & 22.5  \\
				& 100 & 1\,402 & 41.9 & 90  \\ 
				& 500 & 1064 & 25.6 & 67 \\
				& self-tuned & 967 & 16.5 & 18.5  
			\end{tabular}
		\end{center}
	\end{table}		

	}

	\comment{
		For the metrics in Table 2, can the authors also report average values of the three metrics (relative improvements) obtained across the four set point changes considered? The average values would give an idea of the overall control performance. It seems that the overall performance for the settling time would be worse, and only little improvement in the overshoot metric?
	}

	\answer{
		We would like to thank the Reviewer for the comment on the relative improvements. If the values in Table~2 were averaged, it is true that the average settling time would be worse, and the average relative improvement of the overshoot would be very low. However, considering the average of these values would be very strict because of the way how the relative improvements were calculated.
		
		The values in Table~2 are computed subject to the second best setup for each reference step change individually. The negative numbers represent deterioration compared to the best controller setup in the corresponding reference tracking. It means, that the base subject to which is every value calculated, changes. Therefore, averaging the values of the relative improvements would not be fair, as it would not reflect improvement subject to one controller with a constant setup.
		
		To avoid the ``unfair'' evaluation of the relative improvement, we extended Table~2 such that it provides the relative improvements subject to each boundary controller individually, see Table~\ref{tab:improvement}.
		
		\begin{table}[h!]
			\begin{center}
				\caption{Relative improvement of the control performance using the self-tunable explicit MPC controller.}
				\label{tab:improvement}
				\begin{tabular}{c|c|c|c|c} 
					& Comparison subject to the $Q_\mathrm{y}$ setup & $\delta$ SSE\,[\%] & $\delta \sigma_{\mathrm{max}}$\,[\%] & $\delta t_{\epsilon}$\,[\%]  \\
					\hline
					\multirow{2}{*}{ 35\,$^{\circ}$C $\rightarrow$ 45\,$^{\circ}$C } & 1000 & 5 &  55 & 42 \\
					& 100 & 22 &  9 & 24 \\  
					\hline
					\multirow{2}{*}{45\,$^{\circ}$C $\rightarrow$ 50\,$^{\circ}$C } & 1000 & 32 & 59 &$-90$  \\ 
					& 100 & 59 &  18 & 64 \\ 
					\hline
					\multirow{2}{*}{50\,$^{\circ}$C $\rightarrow$ 45\,$^{\circ}$C } & 1000 & 24 &$-30$& 0 \\ 
					& 100 & 53 & 69 & 79 \\
					\hline
					\multirow{2}{*}{45\,$^{\circ}$C $\rightarrow$ 35\,$^{\circ}$C } &  1000 & 6 & 11 & 18 \\
					& 100 & 31 &  61 & 79 \\
					\hline
					\multirow{2}{*}{Average across all steps} &  1000 & 17 & 24 & -7 \\
					& 100 & 41 &  39 & 62  
				\end{tabular}
			\end{center}
		\end{table}
	
			\begin{table}[h!]
		\begin{center}
			\caption{Relative improvement of the control performance using the self-tunable explicit MPC controller.}
			\label{tab:improvement2}
			\begin{tabular}{c|c|c|c|c} 
				& Comparison subject to the $Q_\mathrm{y}$ setup & $\delta$ SSE\,[\%] & $\delta \sigma_{\mathrm{max}}$\,[\%] & $\delta t_{\epsilon}$\,[\%]  \\
				\hline
				\multirow{3}{*}{ 35\,$^{\circ}$C $\rightarrow$ 45\,$^{\circ}$C } & 1000 & 5 &  55 & 42 \\
				& 100 & 22 &  9 & 24 \\ 
				& 500 & -16 & 13 & -27 \\
				\hline
				\multirow{3}{*}{45\,$^{\circ}$C $\rightarrow$ 50\,$^{\circ}$C } & 1000 & 32 & 59 &$-90$  \\ 
				& 100 & 59 &  18 & 64 \\ 
				& 500 & -48 & -5 & -46 \\
				\hline
				\multirow{3}{*}{50\,$^{\circ}$C $\rightarrow$ 45\,$^{\circ}$C } & 1000 & 24 &$-30$& 0 \\ 
				& 100 & 53 & 69 & 79 \\
				& 500 & 12 & -10 & 19 \\
				\hline
				\multirow{3}{*}{45\,$^{\circ}$C $\rightarrow$ 35\,$^{\circ}$C } &  1000 & 6 & 11 & 18 \\
				& 100 & 31 &  61 & 79 \\
				& 500 & 9 & 36 & 72 \\
				\hline
				\multirow{3}{*}{Average across all steps} &  1000 & 17 & 24 & -7 \\
				& 100 & 41 &  39 & 62 \\ 
				& 500 & -11 & 8 &  5
			\end{tabular}
		\end{center}
	\end{table}
	
	
	}
	
	\comment{
		All the three performance metric are related to the controlled temperature variable. Why did the authors not consider any metric related to the control input?	
	}
	
	
	\comment{
		In addition, the authors only summarize the best case performance improvements in the conclusions. This can be a bit misleading to readers since the proposed method also performs worse (significantly) in some parts of the simulations. So summarizing average values would be more appropriate to give a better picture of the overall performance.
	}
	
	
	\comment{
		Based on the responses to my previous comments, it appears that the applicability of the proposed self-tunable explicit MPC method is limited to "small-scale multivariable systems, with decoupled dynamics"? If so, please mention this clearly in the abstract.
	}

	\section*{Answer to Reviewer 2}
	\comment{My comments from the previous round have been fully address. I suggest publication.}
	
	\answer{		
		We thank the Reviewer for reviewing our revised manuscript.		
	}




	

	
	
	\bibliographystyle{plain}
	\bibliography{references}
	
\end{document}


